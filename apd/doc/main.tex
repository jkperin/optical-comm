\documentclass[a4paper]{article}

\usepackage[english]{babel}
\usepackage[utf8]{inputenc}
\usepackage{amsmath}
\usepackage{graphicx}
\usepackage[numbered]{bookmark}
\usepackage[colorinlistoftodos]{todonotes}

\title{APD Analysis and Simulations}

\author{JKP}

\date{\today}

\begin{document}
\maketitle

\section{Modulator}
\subsection{Frequency response}

Second-order system with unit damping and cutoff frequecy $f_c$:
\begin{equation}
	H_{mod}(f) =  \frac{1}{1 + 2jf/f_c - (f/f_c)^2}
\end{equation}
Group delay:
\begin{equation}
	\Delta\tau_g = \frac{2}{2\pi f_c}
\end{equation}

\subsection{Extinction ratio}
\begin{equation}
	r_{ex} = \frac{P_{min}}{P_{max}}
\end{equation}

This relation is enforced in the levels of the PAM signal. Thus, if the levels of a M-PAM signal are $\{a_0, \ldots, a_{M-1}\}$ we have

\begin{equation}
	r_{ex} = \frac{a_0}{a_{M-1}}
\end{equation}

In the case of equally-spaced levels, when a target transmitted power is specified $P_{tx}$, we can determine the power of the lowest level
\begin{equation}
	r_{ex} = \frac{P_{min}}{2P_{tx} - P_{min}} \rightarrow P_{min} = 2P_{tx}\frac{r_{ex}}{1 + r_{ex}}
\end{equation}

Thus the scaled levels $\{a'_0, \ldots, a'_{M-1}\}$ such that the signal has power $P_{tx}$ and extinction ratio $r_{ex}$ can be related to the original levels $\{a_0, \ldots, a_{M-1}\}$ by:
\begin{align} \nonumber
	a'_i &= \frac{2P_{tx} - 2P_{min}}{a_{M-1}}a_i + P_{min} = 2\bigg(P_{tx} - 2P_{tx}\frac{r_{ex}}{1 + r_{ex}}\bigg)\frac{a_i}{a_{M-1}} + P_{min} \\
	&= \frac{2P_{tx}}{a_{M-1}}\bigg(1 - 2\frac{r_{ex}}{1 + r_{ex}}\bigg)a_i + P_{min} = \frac{2P_{tx}}{a_{M-1}}\frac{1-r_{ex}}{1 + r_{ex}}a_i + P_{min} 
\end{align}
where $a_0 = 0$.

For optimized level spacing the extinction ratio is enforced in the optimization process, where at the beginning of the $i$th iteration, the lowest level is set to be $a_0^{(i)} = r_{ex}a_{M-1}^{(i-1)}$

\subsection{RIN}
Modeled as an AWG noise added to the optical power at the transmitter.

The value of RIN is defined as the ratio between the noise power divide by the noise bandwidth and the signal power \cite{agilent-RIN-measurement}: 
\begin{equation}
	RIN = \frac{P_{noise}}{B_{noise}P_{signal}}
\end{equation}

Thus the \textbf{one-sided RIN PSD} and \textbf{RIN variance} at a certain instant are given by
\begin{align}
	& S_{RIN}(t) = RIN\cdot P(t)^2 \\
	& \sigma^2_{RIN}(t) = S_{RIN}(t)\frac{f_{s, sim}}{2}
\end{align}
where $f_{s, sim}$ is the sampling frequency to simulate continuous time.

Output optical power $P(t)$ is given by
\begin{equation}
	P(t) = P_s(t) + w_{RIN}(t)
\end{equation}
where $P_s(t)$ is the signal-only optical power (after modulator frequency response and extinction ratio), and $w_{RIN}(t)\sim\mathcal{N}(0, \sigma^2_{RIN}(t))$.

\subsection{Chirp}
Assuming transient chirp dominant, we have the phase change caused by intensity modulation is given by
\begin{equation}
	\Delta\phi(t) = \frac{\alpha}{2}\ln(P(t))
\end{equation}
where $\alpha$ is the chirp parameter, which is always positive for directly-modulated lasers.

Output electric field of the modulator:
\begin{equation}
	E_{out}(t) = \sqrt{P(t)}e^{j\Delta\phi(t)}
\end{equation}

\section{Fiber}
\subsection{Linear propagation}
\begin{equation}
	E(f, L) = E(f, 0)e^{j\theta}e^{-\frac{1}{2}\frac{att(\lambda)L}{10^4}}
\end{equation}
where $E(f, L)$ is the frequency response of the electric field at distance $L$ in meters; $\theta = -1/2\beta_2(2\pi f)^2L$; $\beta_2 = -\frac{D(\lambda)\lambda^2}{2\pi c}$; and $att(\lambda)$ is the fiber attenuation at wavelength $\lambda$ in dB/km.

For SMF28 the fiber dispersion is specified in terms of the zero-dispersion ($\lambda_0$) wavelength and the dispersion slope ($S_0$):
\begin{equation}
	D(\lambda) = \frac{S_0}{4}\bigg(\lambda - \frac{\lambda_0^4}{\lambda^3}\bigg), \text{for}~1200~\text{nm} < \lambda < 1600~\text{nm}
\end{equation}

\subsection{Small-signal fiber frequency response}
Assuming transient chirp dominant, chromatic dispersion, and attenuation, the small-signal fiber frequency response is given by
\begin{equation}
	H(f) = \frac{P(f, L)}{P(f, 0)} = \Big(\cos(\theta) - \alpha\sin(\theta)\Big)e^{-\frac{att(\lambda)L}{10^4}}
\end{equation}


\section{Using Saddlepoint approximation in the inversion formula of a Moment Generating Function} 

The moment generating function is defined as
\begin{align}
M(s) = E[e^{sx}] &= \begin{cases}
\int_{-\infty}^{\infty} p(x)e^{sx}dx, & \text{continuos} \\
\sum p(x)e^{sx}, & \text{discrete}
\end{cases}
\end{align}
for both continuous and discrete random variables.

The inversion formula is given by
\begin{equation}
p(x) = \frac{1}{2\pi j}\int_{c-j\infty}^{c+j\infty}M(s)e^{-sx}ds= \frac{1}{2\pi j}\int_{c-j\infty}^{c+j\infty}e^{K(s, x)}ds
\end{equation}
Where $K(s,x) = \log M(s) - sx$.

The contour of integration is the straight line in the $s$ plane such that $\mathrm{Re}\{s\} = c$. 

$c$ should be within the interval of convergence for real $s$.

In the saddlepoint approximation $c = \mathrm{Re}\{\hat{s}\}$, where $\frac{\partial}{\partial s} K(s, x)|_{s=\hat{s}} = 0$. Thus, $c$ is the real part of the saddlepoint (of the exponent of the inverse formula). Therefore, $s = \hat{s} + j\omega$. Expanding $K(s,x)$ near $\hat{s}$

\begin{align} \nonumber
K(s, x) &= K(\hat{s}, x) + \frac{(s-\hat{s})^2}{2!}\frac{\partial^2}{\partial s^2}K(s,x)\bigg|_{s=\hat{s}} \\
&= K(\hat{s}, x)  -\frac{\omega^2}{2}K^{\prime\prime}(\hat{s},x)
\end{align}

Applying the saddlepoint approximation in the inversion formula results

\begin{align} \nonumber
p(x) = \frac{e^{K(\hat{s}, x)}}{2\pi }\int_{-\infty}^{\infty}e^{  -\frac{\omega^2}{2}K^{\prime\prime}(\hat{s},x)}d\omega
\approx e^{K(\hat{s},x)}\bigg(\frac{1}{2\pi K^{\prime\prime}(\hat{s},x)}\bigg)^{1/2}
\end{align}
where $K^{\prime\prime}(s,x) = \frac{\partial^2}{\partial s} K(s,x)$.

\subsection{Calculating tail probabilities}
\begin{align}
Q(x) &= P(X \geq x) \\
&= \int_x^{\infty} p(y)dy \\ \nonumber
&= \int_x^{\infty}\frac{1}{2\pi j}\int_{c-j\infty}^{c+j\infty}e^{\ln M(s) - sy}dsdy \\ \nonumber
&= \frac{1}{2\pi j}\int_{c-j\infty}^{c+j\infty}\int_x^{\infty}e^{\ln M(s) - sy}dyds \\ \nonumber
&= \frac{1}{2\pi j}\int_{c-j\infty}^{c+j\infty}\frac{1}{s}e^{\ln M(s) - sx}ds \\ 
&= \frac{1}{2\pi j}\int_{c-j\infty}^{c+j\infty}e^{\ln M(s) -\ln s- sx}ds 
\end{align}

We can define $K(s,x) = \ln\frac{M(s)}{s} - sx$. Applying the saddlepoint approximation with $K(s)$ results in

\begin{equation}
Q(x) \approx e^{K(\hat{s},x)}\bigg(\frac{1}{2\pi K^{\prime\prime}(\hat{s},x)}\bigg)^{1/2}
\end{equation}
where in this case $K^{\prime}(\hat{s},x) = 0$.
Using the saddlepoint approximation in this form might lead to inaccurate results because of the pole at $s = 0$ (i.e., $\hat{s}$ small).

\section{APD}
\subsection{Output signal statistics}

Given an input average optical power $\bar{P}$, the number of $m$ primary electrons is Poisson distributed \cite{personick}: $m \sim \mathrm{Poisson}(\lambda)$, where $\lambda = (RP + I_d)dt/q$, $R$ is the photodiode responsivity; $I_d$ is the dark current; $dt$ is the sampling time of the received signal (sampling here means that the input signal is assumed to occur in discrete times); and $q$ is the electron charge.

The probability generating function (pgf) of the primaries distribution $p_1(m|P)$ is given by
\begin{equation}
g(z) = \sum_{r = 0}^{\infty} p_1(m|P)z^r = e^{\lambda(z-1)}
\end{equation}

Personick \cite{personick-comp-4methos} showed that the PGF of secondary electrons $p_2(n|1)$ (probability of generating $n$ secondary electrons from 1 primary) is
\begin{equation}
M(z) = \sum_{n = 0}^{\infty} p_2(n|1)z^n
\end{equation}
with $z = M[1 + \alpha(M-1)]^{-\beta}, \beta= (1-k)^{-1} > 1, \text{and}~G = (1-\alpha\beta)^{-1}.$

Later Balaban, Fleisher, and Zucker \cite{gain-distribution} showed that $p_2(n|m) \sim M(z)^m$, where $p_2(n|1) \sim M(z)$. That is, the gain of each primary electron is independent.
Therefore the probability of having $n$ output electrons for a given input optical power $P$ is

\begin{equation} \label{output-probability}
p(n|P) = \sum_{m = 0}^{\infty} p_2(n|m)p_1(m|P) =  \sum_{m = 0}^{\infty} p_2(n|m) \frac{\lambda^m e^{\lambda}}{m!}
\end{equation}
whose pgf is given by
\begin{align}
f(z) &= \sum_{k = 0}^{\infty}p(n|P)z^k = \sum_{k=0}^{\infty}\bigg(\sum_{m=0}^{\infty}p_1(m|P)p_2(n|m)\bigg)z^k \\ \nonumber
& = \sum_{m=0}^{\infty}p_1(m|P)\bigg(\sum_{k=0}^{\infty}p_2(n|m)z^k\bigg) \\ \nonumber
& = \sum_{m=0}^{\infty}p_1(m|P)M(z)^m \\
& = g(M(z)) = e^{\lambda(M(z)-1)}
\end{align}

From \cite{ber-saddlepoint-approx}, the pmf and its tails are given by
\begin{align}
p(n|P) = \frac{1}{2\pi j}\oint_C z^{-(n+1)}f(z)dz \\
p(N \geq n| P) = \frac{1}{2\pi j}\oint_{C+} \frac{z^{-n}}{z-1}f(z)dz \\
p(N \leq n| P) = \frac{1}{2\pi j}\oint_{C-} \frac{z^{-n}}{1-z}f(z)dz
\end{align}

We can apply the saddlepoint approximation to the inversion formulas: 

\subsection{pdf and tail probabilities including thermal noise and using the saddlepoint approximation}
\subsubsection{pdf}
The MGF of the signal including thermal noise with variance $\sigma^2$ is given by
\begin{equation}
\Phi (s) = \exp\Big(\lambda (M(e^s) - 1) + \frac{1}{2}\sigma^2s^2\Big)
\end{equation}

Thus, from the inversion equation we can write
\begin{align}
& K(s,x) = \log (\Phi(s)) - sx = \lambda (M(e^s) - 1) + \frac{1}{2}\sigma^2s^2 - sx \\
&\frac{\partial K(s,x)}{\partial s} = \lambda \frac{d}{ds}M(e^s) + \sigma^2s - x \\
&\frac{\partial^2 K(s,x)}{\partial s^2} = \lambda \frac{d^2}{ds^2}M(e^s) + \sigma^2
\end{align}

The main difficulty in evaluating the derivatives of $K(s,x)$ is in the fact that $M(e^s)$ has no explicit relation; it is only given in terms of the implicit relation
\begin{equation}
e^s = z = M(1 + \alpha(M-1))^{-\beta}
\end{equation}
where $\beta = (1 - k)^{-1}$, and $G = (1-\alpha\beta)^{-1}$.

Thus, we can write
\begin{align}
&\frac{d}{ds}M(s) = \frac{M(1 + \alpha (M-1))}{1 + \alpha(M-1) - \alpha\beta M} = \frac{p(s)}{q(s)}\\
&\frac{d^2}{ds^2}M(s) = \frac{p'q - pq'}{q^2}
\end{align}
where $p'(s) = M'(1 + \alpha(2M-1))$, and $q'(s) = \alpha M'(1 - \beta)$.

\begin{align}
&\frac{\partial K(s,x)}{\partial s} = \lambda \frac{M(1 + \alpha (M-1))}{1 + \alpha(M-1) - \alpha\beta M} + \sigma^2\log\Big(M(1 + \alpha(M-1))^{-\beta}\Big) - x \\
\end{align}

\subsubsection{tail}
Similarly for the tail probabilities, the MGF of the signal including thermal noise with variance $\sigma^2$ is given by
\begin{equation}
\Phi (s) = \exp\Big(\lambda (M(e^s) - 1) + \frac{1}{2}\sigma^2s^2\Big)
\end{equation}

Thus, from the inversion equation we can write
\begin{align}
& K(s,x) = \log \Big(\frac{\Phi(s)}{s}\Big) - sx = \lambda (M(e^s) - 1) + \frac{1}{2}\sigma^2s^2 - sx - \log(|s|) \\
&\frac{\partial K(s,x)}{\partial s} = \lambda \frac{d}{ds}M(e^s) + \sigma^2s - x - \frac{1}{s} \\
&\frac{\partial^2 K(s,x)}{\partial s^2} = \lambda \frac{d^2}{ds^2}M(e^s) + \sigma^2 + \frac{1}{s^2} 
\end{align}

The derivatives of $M(e^s)$ remain the same, and we can write

\begin{align}
&\frac{\partial K(s,x)}{\partial s} = \lambda \frac{M(1 + \alpha (M-1))}{1 + \alpha(M-1) - \alpha\beta M} + \sigma^2\log\Big(M(1 + \alpha(M-1))^{-\beta}\Big) - x -\frac{1}{s}
\end{align}

\subsection{Gaussian approximation}
The Gaussian approximation is equivalent to modeling equation \eqref{output-probability} as a Gaussian random variable \cite{personick-comp-4methos}. This approximation is fairly accurate because even for small power levels such as $-30$ dB, we have $\lambda = 125$, assuming a rate of 50 GHz.

The current is equal to $I = \frac{nq}{dt} = \frac{n}{\lambda}(RP + I_d)$, where $n \sim p(n|P)$ as in \eqref{output-probability}. Thus, 
\begin{align}
& E(I) = RP + I_d \\
& \mathrm{Var}(I) = G^2F_A(G)\cdot 2q(RP_{rec} + I_d)1/dt
\end{align}
where $F_A(G)G^2$ corresponds to the mean square value of the random gain \cite{personick-comp-4methos}.

\textbf{Excess noise factor}
\begin{equation}
F_A(G) = k_AG + (1-k_A)\Big(2 - \frac{1}{G}\Big)
\end{equation}

\textbf{One-sided shot noise psd}
\begin{equation}
S_{shot} = G^2F_A(G)\cdot 2q(RP_{rec} + I_d)
\end{equation}

\textbf{APD noise figure:} The noise figure is defined as the $F_n = SNR_{in}(dB) - SNR_{out}(dB)$, where $SNR_{in}$ is defined as the SNR assuming only shot noise in a ideal photo counter, and $SNR_{out}(dB)$ is the SNR assuming the detection with the APD.

\underline{The excess noise factor is equal to the APD noise figure}, where the noise figure is defined as the ratio between the $SNR_{in}$ assuming detection with a photodiode with no dark current, and $SNR_{out}$ assuming detection with an APD with no dark current.
\begin{align}
F_n &= \frac{SNR_{in}}{SNR_{out}} = \frac{\frac{(R\bar{P})^2}{2qR\bar{P}\Delta f}}{\frac{(GR\bar{P})^2}{G^2F_A(G)\cdot 2qR\bar{P}\Delta f}} = F_A(G)
\end{align}

\textbf{BER with equally-spaced levels}
\begin{equation}
BER = \frac{1}{M}Q\bigg(\frac{\Delta P}{\sqrt{\sigma_{0}^2 + \sigma_T^2}}\bigg) + \frac{2}{M}Q\bigg(\frac{\Delta P}{\sqrt{\sigma_{1}^2 + \sigma_T^2}}\bigg) + \ldots + \frac{1}{M}Q\bigg(\frac{\Delta P}{\sqrt{\sigma_{M-1}^2 + \sigma_T^2}}\bigg)
\end{equation}
where $\sigma_{i}^2$ signal-dependent noise variance of the $i$th level, $\sigma_{T}$ is thermal noise variance or any other signal-independent noise, $\Delta P = \frac{2\bar{P}}{M-1}\frac{1-r_{ex}}{1 + r_{ex}}$ is the level spacing.

\subsection{Power Penalty vs Noise Figure}

Shot noise is dominant, and the BER is determined by the argument of the Q-function
\begin{equation}
\frac{\Delta P}{\sigma_{shot}} \propto \frac{GP}{\sqrt{G^2F_A(G)P}} = \sqrt{\frac{P}{F_A(G)}}
\end{equation}
Thus, 1-dB increase in the noise figure corresponds in 1-dB power penalty. However, for APD the noise figure depends on the operation point i.e., Gain.

\begin{thebibliography}{9}
\bibitem{personick-comp-4methos} Personick, S. D., Balaban, P., Bobsin, J. H., and Kumar, P.; ``R.A Detailed Comparison of Four Approaches to the Calculation of the Sensitivity of Optical Fiber System Receivers,'' \emph{IEEE Transactions on Communications}, 1976.

\bibitem{gain-distribution} Balaban, P., Fleischer, P. E., and Zucker, H.; ``The Probability Distribution of Gains in Avalanche Photodiodes,'' \emph{IEEE Transactions on Electron Devices}, 1976.

\bibitem{ber-saddlepoint-approx} Carl Helstrom; ``Performance Analysis of Optical Receivers by the Saddlepoint Approximation,'' \emph{IEEE Transactions on Communications}, 1979.

\bibitem{personick} Personick, S. D.; ``Statistics of a General Class of Avalanche Detectors With Applications to Optical Communication,'' \emph{The Bell SystemTechnical Journal}, 1971.

\bibitem{agilent-RIN-measurement} Product Note 86100-7; ``Digital Communication Analyzer (DCA), Measure Relative Intensity Noise (RIN),'' Agilent Technologies. 

\end{thebibliography}

\end{document}