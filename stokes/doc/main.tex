\documentclass[a4paper]{article}

%% Language and font encodings
\usepackage[english]{babel}
\usepackage[utf8x]{inputenc}
\usepackage[T1]{fontenc}

%% Sets page size and margins
\usepackage[a4paper,top=3cm,bottom=2cm,left=3cm,right=3cm,marginparwidth=1.75cm]{geometry}

%% Useful packages
\usepackage{amsmath}
\usepackage{amsfonts}
\usepackage{amssymb}

\usepackage{graphicx}
\usepackage[colorinlistoftodos]{todonotes}
\usepackage[colorlinks=true, allcolors=blue]{hyperref}

\DeclareMathOperator{\E}{\mathbb{E}}

\title{Stokes Vector Receiver}
\author{JKP}

\begin{document}
\maketitle

The electric field at the output of a single-mode fiber is related to the input electric field by the Jones matrix:
\begin{equation}
\begin{bmatrix}
E_X\\
E_Y
\end{bmatrix} =\begin{bmatrix}
a & -b \\
b^* & a^*
\end{bmatrix}
\begin{bmatrix}
E_X\\
E_Y
\end{bmatrix}
\end{equation}

For the Stokes vector receiver, it is more convinient to write the input and output vectors in a 4-dimensional space.
\begin{equation}
\begin{bmatrix}
|E_X|^2 \\
|E_Y|^2 \\
\mathrm{Re}\{E_XE_Y^*\} \\
\mathrm{Im}\{E_XE_Y^*\} 
\end{bmatrix} = M\begin{bmatrix}
|E_X|^2 \\
|E_Y|^2 \\
\mathrm{Re}\{E_XE_Y^*\} \\
\mathrm{Im}\{E_XE_Y^*\} 
\end{bmatrix},
\end{equation}
where $M$ is given by
\begin{equation}
M = \begin{bmatrix}
|a|^2 & |b|^2 & -2\mathrm{Re}\{ab^*\} & 2\mathrm{Im}\{ab^*\} \\
|b|^2 & |a|^2 & 2\mathrm{Re}\{ab^*\} & -2\mathrm{Im}\{ab^*\} \\
\mathrm{Re}\{ab\} & -\mathrm{Re}\{ab\} & \mathrm{Re}\{a^2\}-\mathrm{Re}\{b^2\} & -\mathrm{Im}\{a^2\}-\mathrm{Im}\{b^2\} \\
\mathrm{Im}\{ab\} & -\mathrm{Im}\{ab\} & \mathrm{Im}\{a^2\}-\mathrm{Im}\{b^2\} & \mathrm{Re}\{a^2\}+\mathrm{Re}\{b^2\}
\end{bmatrix}.
\end{equation}
Note that $M$ is no longer an unitary matrix, and hence the noises will scale differently.

%     M = [abs(a)^2, abs(b)^2, -2*real(a*b'), 2*imag(a*b');...
%         abs(b)^2, abs(a)^2,  2*real(a*b'), -2*imag(a*b');...
%         real(a*b), -real(a*b),  real(a^2) - real(b^2), -imag(a^2) - imag(b^2);...
%         imag(a*b), -imag(a*b), imag(a^2) - imag(b^2), real(a^2) + real(b^2)];

Due to the splitters in the receiver, the detected vector $Y$ is
\begin{equation}
Y= \begin{bmatrix}
|E_X|^2 \\
|E_Y|^2 \\
\mathrm{Re}\{E_XE_Y^*\} \\
\mathrm{Im}\{E_XE_Y^*\}
\end{bmatrix} + \bf{n}
\end{equation}
where $\bf{n}$ is a random vector whose correlation matrix is
\begin{equation}
\E\{{\bf nn^H}\} = 4\sigma^2I,
\end{equation}
for a thermal-noise limited receiver, and 
\begin{equation}
\E\{{\bf nn^H}\} = P_s\sigma^2\mathrm{diag}([2, 2, 1, 1]), 
\end{equation}
where $P_s$ is the total received power, and $P_s = 1/2\E(|E_X|^2) = 1/2\E(|E_Y|^2)$. For an amplified system, $\sigma^2 = S_{sp}\Delta f$ is the ASE variance per real dimension.

To recover the transmitted state, the MIMO signal processing at the receiver will find a matrix $M^{\dagger}$ to invert the channel. 


\section{Polarization intensity signals}

The signal detected by the first PD is given by
\begin{equation}
I_1 = \Big|\frac{E_X}{\sqrt{2}}\Big|^2 = \frac{1}{2}|E_X|^2
\end{equation}
Assuming amplified noise, and ignoring spont-spont noise:
\begin{align} \nonumber
I_1 &= \frac{1}{2}|E_X + n_X|^2 = \frac{1}{2}\Big(|E_X|^2 + |n_X|^2 + E_Xn_X^* + E_X^*n_X\Big) \\ \nonumber
&\approx\frac{1}{2}\Big(|E_X|^2 + 2\mathrm{Re}(E_Xn_X^*)\Big)  \\
&=\frac{1}{2}\Big(|E_X|^2 + \sqrt{4P_X}n_1\Big)
\end{align}
where $n_1 \sim\mathcal{N}(0, \sigma^2)$, where $\sigma^2$ is the ASE variance per real dimension.

Similarly, for the $Y$ polarization:
\begin{align} \nonumber
I_2 \approx \frac{1}{2}\Big(|E_Y|^2 + \sqrt{4P_Y}n_2\Big)
\end{align}
where $n_2 \sim\mathcal{N}(0, \sigma^2)$, where $\sigma^2$ is the ASE variance per real dimension. $n_1$ is independent of $n_2$, since each noise derives from the noise in the X, and Y polarizations respectively.

\section{Polarization beating signals}

The 90-degree hybrid has the following transfer matrix:
\begin{equation}
\begin{bmatrix}
E_1 \\
E_2 \\
E_3 \\
E_4 
\end{bmatrix} = \frac{1}{2\sqrt{2}}\begin{bmatrix}
1 & 1 \\
j & -j \\
j & -1 \\
-1 & j 
\end{bmatrix}\begin{bmatrix}
E_X \\
E_Y
\end{bmatrix}
\end{equation}

Hence the detected current of the first balanced photodiodes is given by
\begin{align}
I_3 = |E_1|^2 - |E_2|^2 = \frac{1}{2}\mathrm{Re}\{E_XE_Y^*\}
\end{align}

Including amplified noise:
\begin{align} \nonumber
I_2 &= \frac{1}{2}\mathrm{Re}\{(E_X + n_X)(E_Y + n_Y)^*\} \\ \nonumber
&\approx \frac{1}{2}\mathrm{Re}\{E_XE_Y^*\} + \frac{1}{2}\mathrm{Re}\{E_Xn_Y^* + E_Y^*n_X\} \\
&\approx \frac{1}{2}\mathrm{Re}\{E_XE_Y^*\} + \frac{1}{2}\sqrt{P_X + P_Y}n_2
\end{align}
where the noise approximation follows from
\begin{align}
\E(\mathrm{Re}\{E_Xn_Y^* + E_Y^*n_X\}^2) = \frac{1}{2}\E(|E_Xn_Y^* + E_Y^*n_X|^2) = (P_X + P_Y)\sigma^2
\end{align}
since $\E(|n_Y|^2) = \E(|n_X|^2) = 2\sigma^2$, since $n_X$ and $n_Y$ are circularly Gaussian random variables.

\end{document}