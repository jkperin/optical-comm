\documentclass[a4paper]{article}

\usepackage[english]{babel}
\usepackage[utf8]{inputenc}
\usepackage{amsmath}
\usepackage{graphicx}
\usepackage[numbered]{bookmark}
\usepackage[colorinlistoftodos]{todonotes}
\usepackage{algorithm}
\usepackage{algpseudocode}
\usepackage{pifont}
\usepackage{tikz}
\usepackage{pgfplots}
\usepackage{bm}

\DeclareGraphicsExtensions{.eps,.pdf,.png,.tikz}
\graphicspath{{figs/}}

\title{Modeling of Erbium Doped Fiber Amplifiers}

\author{JKP}

\date{\today}

\begin{document}
\maketitle

\section{Numerical Modeling of Spectral Properties}

Following the development of \cite{Giles1991} for a two-level model, the spectral properties of an EDFA are governed by the differential equation
\begin{align}
\frac{dP_k}{dz} = u_k(\alpha_k + g_k)\frac{\bar{n}_2}{\bar{n}_t}P_k(z) + u_kg_k\frac{\bar{n}_2}{\bar{n}_t}mh\nu_k\Delta\nu_k - u_k(\alpha_k + l_k)P_k
\end{align}
where
\begin{equation}
\frac{\bar{n}_2}{\bar{n}_t} = \frac{\sum_k \frac{P_k(z)\alpha_k}{h\nu_k\xi}}{1 + \sum_k \frac{P_k(z)(\alpha_k + g_k)}{h\nu_k\xi}}
\end{equation}

\begin{table}[h]
	\caption{Parameters definition.}
	\label{tab:param}
	\centering
	\begin{tabular}{r|l}
		\hline
		$P_k(z)$ & power of the $k$th pump/signal at position $z$ \\
		$u_k$ & direction of $k$th pump/signal. $u_k = 1$, for forward, and $u_k = -1$ for backward \\
		$\alpha_k$ & absorption coefficient \\
		$g_k$ & gain coefficient \\
		$l_k$ & excess loss accounts for additional fiber loss \\
		$\xi$ & saturation power parameter $\mathrm{m}^{-1}$ \\
		$\nu_k$ & wavelength of the $k$ pump/signal \\
		$m$ & number of supported modes (m = 2 for SMF)\\
		$b_{eff}$ & equivalent radius of doped region \\
		$\bar{n}_t$ & average density of the metastable state \\
		$\Delta\nu_k$ & resolution of ASE spectrum \\
		$\tau$ & metastable lifetime (10 ms) \\
		$h$ & Planck's constant \\
		\hline
	\end{tabular}
\end{table}

The parameter $\xi$ is defined as $\xi = \pi b_{eff}^2\bar{n}_t/\tau$ i.e., the ratio of the linear density (m$^{-1}$) of ions to the metastable lifetime. It can be determined from measurement of the fiber saturation power 
\begin{equation}
\xi = P^{sat}_k\frac{\alpha_k+g_k}{h\nu_k}
\end{equation}

In \cite{Giles1991}, $\xi = 1.5\times10^{15}~\mathrm{m}^{-1}$ or  $\xi = 4.2\times10^{15}~\mathrm{m}^{-1}$.

\bibliographystyle{plain}
\bibliography{bib}

\end{document}