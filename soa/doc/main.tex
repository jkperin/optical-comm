\documentclass[a4paper]{article}

\usepackage[english]{babel}
\usepackage[utf8]{inputenc}
\usepackage{amsmath}
\usepackage{graphicx}
\usepackage[numbered]{bookmark}
\usepackage[colorinlistoftodos]{todonotes}

\title{SOA Analysis and Simulations}

\author{JKP}

\date{\today}

\begin{document}
\maketitle

\section{Modulator}
\subsection{Frequency response}

Second-order system with unit damping and cutoff frequecy $f_c$:
\begin{equation}
	H_{mod}(f) =  \frac{1}{1 + 2jf/f_c - (f/f_c)^2}
\end{equation}
Group delay:
\begin{equation}
	\Delta\tau_g = \frac{2}{2\pi f_c}
\end{equation}

\subsection{Extinction ratio}
\begin{equation}
	r_{ex} = \frac{P_{min}}{P_{max}}
\end{equation}

This relation is enforced in the levels of the PAM signal. Thus, if the levels of a M-PAM signal are $\{a_0, \ldots, a_{M-1}\}$ we have

\begin{equation}
	r_{ex} = \frac{a_0}{a_{M-1}}
\end{equation}

In the case of equally-spaced levels, when a target transmitted power is specified $P_{tx}$, we can determine the power of the lowest level
\begin{equation}
	r_{ex} = \frac{P_{min}}{2P_{tx} - P_{min}} \rightarrow P_{min} = 2P_{tx}\frac{r_{ex}}{1 + r_{ex}}
\end{equation}

Thus the scaled levels $\{a'_0, \ldots, a'_{M-1}\}$ such that the signal has power $P_{tx}$ and extinction ratio $r_{ex}$ can be related to the original levels $\{a_0, \ldots, a_{M-1}\}$ by:
\begin{align} \nonumber
	a'_i &= \frac{2P_{tx} - 2P_{min}}{a_{M-1}}a_i + P_{min} = 2\bigg(P_{tx} - 2P_{tx}\frac{r_{ex}}{1 + r_{ex}}\bigg)\frac{a_i}{a_{M-1}} + P_{min} \\
	&= \frac{2P_{tx}}{a_{M-1}}\bigg(1 - 2\frac{r_{ex}}{1 + r_{ex}}\bigg)a_i + P_{min} = \frac{2P_{tx}}{a_{M-1}}\frac{1-r_{ex}}{1 + r_{ex}}a_i + P_{min} 
\end{align}
where $a_0 = 0$.

For optimized level spacing the extinction ratio is enforced in the optimization process, where at the beginning of the $i$th iteration, the lowest level is set to be $a_0^{(i)} = r_{ex}a_{M-1}^{(i-1)}$

\subsection{RIN}
Modeled as an AWG noise added to the optical power at the transmitter.

The value of RIN is defined as the ratio between the noise power divide by the noise bandwidth and the signal power \cite{agilent-RIN-measurement}: 
\begin{equation}
	RIN = \frac{P_{noise}}{B_{noise}P_{signal}}
\end{equation}

Thus the \textbf{one-sided RIN PSD} and \textbf{RIN variance} at a certain instant are given by
\begin{align}
	& S_{RIN}(t) = RIN\cdot P(t)^2 \\
	& \sigma^2_{RIN}(t) = S_{RIN}(t)\frac{f_{s, sim}}{2}
\end{align}
where $f_{s, sim}$ is the sampling frequency to simulate continuous time.

Output optical power $P(t)$ is given by
\begin{equation}
	P(t) = P_s(t) + w_{RIN}(t)
\end{equation}
where $P_s(t)$ is the signal-only optical power (after modulator frequency response and extinction ratio), and $w_{RIN}(t)\sim\mathcal{N}(0, \sigma^2_{RIN}(t))$.

\subsection{Chirp}
Assuming transient chirp dominant, we have the phase change caused by intensity modulation is given by
\begin{equation}
	\Delta\phi(t) = \frac{\alpha}{2}\ln(P(t))
\end{equation}
where $\alpha$ is the chirp parameter, which is always positive for directly-modulated lasers.

Output electric field of the modulator:
\begin{equation}
	E_{out}(t) = \sqrt{P(t)}e^{j\Delta\phi(t)}
\end{equation}

\section{Fiber}
\subsection{Linear propagation}
\begin{equation}
	E(f, L) = E(f, 0)e^{j\theta}e^{-\frac{1}{2}\frac{att(\lambda)L}{10^4}}
\end{equation}
where $E(f, L)$ is the frequency response of the electric field at distance $L$ in meters; $\theta = -1/2\beta_2(2\pi f)^2L$; $\beta_2 = -\frac{D(\lambda)\lambda^2}{2\pi c}$; and $att(\lambda)$ is the fiber attenuation at wavelength $\lambda$ in dB/km.

For SMF28 the fiber dispersion is specified in terms of the zero-dispersion ($\lambda_0$) wavelength and the dispersion slope ($S_0$):
\begin{equation}
	D(\lambda) = \frac{S_0}{4}\bigg(\lambda - \frac{\lambda_0^4}{\lambda^3}\bigg), \text{for}~1200~\text{nm} < \lambda < 1600~\text{nm}
\end{equation}

\subsection{Small-signal fiber frequency response}
Assuming transient chirp dominant, chromatic dispersion, and attenuation, the small-signal fiber frequency response is given by
\begin{equation}
	H(f) = \frac{P(f, L)}{P(f, 0)} = \Big(\cos(\theta) - \alpha\sin(\theta)\Big)e^{-\frac{att(\lambda)L}{10^4}}
\end{equation}


\section{Using Saddlepoint approximation in the inversion formula of a Moment Generating Function} 

The moment generating function is defined as
\begin{align}
M(s) = E[e^{sx}] &= \begin{cases}
\int_{-\infty}^{\infty} p(x)e^{sx}dx, & \text{continuos} \\
\sum p(x)e^{sx}, & \text{discrete}
\end{cases}
\end{align}
for both continuous and discrete random variables.

The inversion formula is given by
\begin{equation}
p(x) = \frac{1}{2\pi j}\int_{c-j\infty}^{c+j\infty}M(s)e^{-sx}ds= \frac{1}{2\pi j}\int_{c-j\infty}^{c+j\infty}e^{K(s, x)}ds
\end{equation}
Where $K(s,x) = \log M(s) - sx$.

The contour of integration is the straight line in the $s$ plane such that $\mathrm{Re}\{s\} = c$. 

$c$ should be within the interval of convergence for real $s$.

In the saddlepoint approximation $c = \mathrm{Re}\{\hat{s}\}$, where $\frac{\partial}{\partial s} K(s, x)|_{s=\hat{s}} = 0$. Thus, $c$ is the real part of the saddlepoint (of the exponent of the inverse formula). Therefore, $s = \hat{s} + j\omega$. Expanding $K(s,x)$ near $\hat{s}$

\begin{align} \nonumber
K(s, x) &= K(\hat{s}, x) + \frac{(s-\hat{s})^2}{2!}\frac{\partial^2}{\partial s^2}K(s,x)\bigg|_{s=\hat{s}} \\
&= K(\hat{s}, x)  -\frac{\omega^2}{2}K^{\prime\prime}(\hat{s},x)
\end{align}

Applying the saddlepoint approximation in the inversion formula results

\begin{align} \nonumber
p(x) = \frac{e^{K(\hat{s}, x)}}{2\pi }\int_{-\infty}^{\infty}e^{  -\frac{\omega^2}{2}K^{\prime\prime}(\hat{s},x)}d\omega
\approx e^{K(\hat{s},x)}\bigg(\frac{1}{2\pi K^{\prime\prime}(\hat{s},x)}\bigg)^{1/2}
\end{align}
where $K^{\prime\prime}(s,x) = \frac{\partial^2}{\partial s} K(s,x)$.

\subsection{Calculating tail probabilities}
\begin{align}
Q(x) &= P(X \geq x) \\
&= \int_x^{\infty} p(y)dy \\ \nonumber
&= \int_x^{\infty}\frac{1}{2\pi j}\int_{c-j\infty}^{c+j\infty}e^{\ln M(s) - sy}dsdy \\ \nonumber
&= \frac{1}{2\pi j}\int_{c-j\infty}^{c+j\infty}\int_x^{\infty}e^{\ln M(s) - sy}dyds \\ \nonumber
&= \frac{1}{2\pi j}\int_{c-j\infty}^{c+j\infty}\frac{1}{s}e^{\ln M(s) - sx}ds \\ 
&= \frac{1}{2\pi j}\int_{c-j\infty}^{c+j\infty}e^{\ln M(s) -\ln s- sx}ds 
\end{align}

We can define $K(s,x) = \ln\frac{M(s)}{s} - sx$. Applying the saddlepoint approximation with $K(s)$ results in

\begin{equation}
Q(x) \approx e^{K(\hat{s},x)}\bigg(\frac{1}{2\pi K^{\prime\prime}(\hat{s},x)}\bigg)^{1/2}
\end{equation}
where in this case $K^{\prime}(\hat{s},x) = 0$.
Using the saddlepoint approximation in this form might lead to inaccurate results because of the pole at $s = 0$ (i.e., $\hat{s}$ small).

\section{ASE -- Gaussian Approximation}

\textbf{Reference: EE348 notes and Agrawal}

\textbf{One-sided ASE noise psd per polarization}
\begin{equation}
S_{sp}(\nu) = (G-1)n_{sp}h\nu
\end{equation}

Noise figure
\begin{equation}
F_n = 2n_{sp}\frac{G-1}{G} \approx 2n_{sp}
\end{equation}
The input coupling $f$ reduces the noise figure i.e., $F_n' = F_n/f$. This is a problem in SOAs and thus they present higher noise figure.

Signal is in one polarization and noise may appear in two polarization with same power.
\begin{align} \nonumber
I(t) &= R|\sqrt{G}E_s(t) + E_n(t)|^2 + |E_n(t)|^2 \\
& = R\Big(G|E_s(t)|^2 + \sqrt{G}E_s(t)E_n^*(t) + \sqrt{G}E_s^*(t)E_n(t) + 2|E_n(t)|^2\Big)
\end{align}
If a polarizer is used, then the noise component on the orthogonal polarization is avoided. More generally we can write
\begin{equation}
I(t) = R\Big(G|E_s(t)|^2 + \sqrt{G}E_s(t)E_n^*(t) + \sqrt{G}E_s^*(t)E_n(t) + N_{pol}|E_n(t)|^2\Big)
\end{equation}
where $N_{pol}$ is the number of noise polarizations. Number of noise polarizations i.e., use of polarizer, doesn't affect signal-spontaneous beat noise.

\subsection{Signal-spontaneous beat noise}
Signal-spontaneous beat noise has a bilinear term of form $\sqrt{G}E_s(t)E_n(t)$. Assuming signal is real and CW over a certain interval, which is consistent with a PAM signal:

\begin{align} \nonumber
I(t) = RG|E_s(t)|^2 + 2\sqrt{G}RE_s(t)\mathrm{Re}\{E_n(t)\} + RP|E_n(t)|^2
\end{align}

Therefore, the signal-spontaneous beat noise has variance
\begin{equation}
\sigma^2_{sig-sp, k} = 4R^2G\bar{P}_{s,k}S_{sp}\Delta f = 2R^2G\bar{P}_sN_0\Delta f
\end{equation}
where $\bar{P}_{s,k}$ is the average signal power at the $k$th level, and $S_{sp} = 2N_0$, where $N_0$ is the equivalent one-sided baseband ASE PSD.

The signal-spontaneous beat noise, which is dominant in well-designed receivers, is not affected by the number of noise polarizations i.e., whether a polarizer is used. Furthermore, it's independent of the band-pass optical filter bandwidth.

\subsection{Spontaneous-spontaneous beat noise}

Spontaneous-spontaneous beat noise PSD is not white; it decays linearly with frequency. At DC the PSD is $2N_{pol}R^2S_{sp}^2\Delta\nu_{opt}$. However, assuming $\Delta f << \Delta\nu_{opt}/2$ we can write

\begin{equation}
\sigma^2_{sp-sp} = 2N_{pol}R^2S_{sp}^2\Delta\nu_{opt}\Delta f = \frac{1}{2}N_{pol}R^2N_0^2\Delta\nu_{opt}\Delta f 
\end{equation}

\section{ASE -- Karhunen-Loéve Series Expansion for MGF Calculation} 

\subsection{Definitions}
\begin{itemize}
	\item $x(t)$ signal component after amplifier;
	\item $w(t)$ ASE noise;
	\item $N_0$ baseband equivalent ASE noise PSD i.e., $N_0 = 2S_{sp}$.
	\item $e(t) = x(t) + w(t)$ electrical field after amplifier;
	\item $h_o(t) \leftrightarrow H_o(f)$ optical bandpass filter;
	\item $h_e(t) \leftrightarrow H_e(f)$ electrical baseband filter;
	\item $e_o(t) = (x(t) + w(t))\ast h_o(t) = s(t) + n(t)$ electric field after  optical bandpass filter;
	\item $y(t) = |e_o(t)|^2\ast h_e(t)$ electric signal after filtering (doesn't include thermal noise);
\end{itemize}

\subsection{Karhunen-Loéve Series Expansion in the Frequency Domain -- Less accurate and requires more eigenvalues}
Electric signal after filtering (doesn't include thermal noise) as a function of the Frequency-domain filter responses
\begin{align} \nonumber
y(t) &= |e(t)\ast h_o(t)|^2\ast h_e(t) \\ \nonumber
& = \mathcal{F}^{-1}\{[E(f)H_o(f)\ast E(-f)^*H_o(-f)^*]H_e(f)\} \\
& = \iint_{-\infty}^{\infty} E(f_1)E^*(f_2)K(f_1, f_2)e^{-j2\pi(f_1-f_2)t}df_1df_2
\end{align}
where $K(f_1, f_2) = H_o(f_1)H_e(f_1-f_2)H_o^*(f_2)$ is obtained simply by change of variables.

The eigenfunctions of the Fredholm integral equation form a complete set of orthonormal basis functions, and the corresponding eigenvalues $\lambda_n$ are real and ordered as $\lambda_1 \geq \lambda_2 \geq \ldots$.
\begin{equation}
\int_{-\infty}^{\infty} K(f_1, f_2)\phi(f_1)df_1 = \lambda\phi(f_2)
\end{equation}

Given the completeness of the eigenfunctions we can use the expansion:
\begin{equation}
E(f)e^{j2\pi ft} = \sum_{n=1}^{\infty}e_n(t)\phi_n(f)
\end{equation}
where 
\begin{equation}
e_n(t) = x_n(t) + w_n(t) = \int_{-\infty}^{\infty}E(f)\phi^*_n(f)e^{j2\pi ft}df
\end{equation}

Using this expansion to solve for $y(t)$ results in
\begin{equation}
y(t) = \sum_{n=1}^{\infty}\lambda_n|x_n(t) + w_n(t)|^2
\end{equation}
The coefficients $w_n(t)$ are statistically independent and circularly symmetric complex random variables with zero mean and variance equal to the ASE spectral density per polarization mode $N_0$.
Therefore, $y(t)$ is a weighted sum of non-central $\chi^2$ independent random variables.

\subsubsection{Evaluating $\lambda_n$ and $x_n$}
Performing the integration in a band $[-F, F]$ where the optical filter $H_o(f)$ is significant results
\begin{equation}
\int_{-\infty}^{\infty} K(f_1, f_2)\phi(f_1)df_1 \approx \int_{-F}^{F} K(f_1, f_2)\phi(f_1)df_1 \approx \lambda\phi(f_2)
\end{equation}
This can be approximated by the \textbf{Gauss-Legendre quadrature rule}
\begin{equation}
\sum_{m=1}^{M_e}w_mK(\nu_m, f_2)\phi_n(\nu_m) \approx \lambda_n\phi_n(f_2)
\end{equation}
where the weights $w_m$ are always positive.

Making $f_2 = \nu_n, n = 1,\ldots, M_e$
\begin{equation}
\sum_{m=1}^{M_e}w_mK(\nu_m, \nu_n)\phi_n(\nu_m) \approx \lambda_n\phi_n(\nu_n), n = 1,\ldots, M_e
\end{equation}
In matrix notation:
\begin{equation} \label{eq:matrix-form}
\mathcal{K}\mathcal{W}\Phi = \Phi\Lambda
\end{equation}
where 
\begin{align}
\mathcal{K}_{ij} = K(\nu_i, \nu_j) \\
\mathcal{W}_{ij} = w_i\delta_{ij} \\
\Phi_{ij} = \phi_j(\nu_i) \\
\Lambda_{ij} = \lambda_i\delta_{ij}
\end{align}
The columns of $\Phi$ are the eigenvectors of $\mathcal{K}\mathcal{W}$. But $\mathcal{K}\mathcal{W}$ is not Hermitian, so instead we can write $A = \mathcal{W}^{1/2}\mathcal{K}\mathcal{W}^{1/2}$, which is Hermitian. $\mathcal{W}^{1/2} = \mathrm{diag}\{\sqrt{w_1},\ldots,\sqrt{w_{M_e}}\}$. So instead of solving the eigenvector problem for \eqref{eq:matrix-form} it's more convenient to solve it for

\begin{equation} \label{eq:matrix-form}
\mathcal{W}^{1/2}\mathcal{K}\mathcal{W}\Phi = \mathcal{W}^{1/2}\Phi\Lambda = AB = B\Lambda
\end{equation}
where $A = \mathcal{W}^{1/2}\mathcal{K}\mathcal{W}^{1/2}$, and $B = \mathcal{W}^{1/2}\Phi$.

Once the eigenvectors $B$ are determine we can solve for $\Phi = \mathcal{W}^{-1/2}B$.

Knowing the sampled $\phi_n(\nu_m)$, we can interpolate it, and finally calculate $x_n$ by:
\begin{equation}
x_n(t) = \int_{-\infty}^{\infty}X(f)\phi_n^*(f)e^{j2\pi ft}df
\end{equation}

\subsection{Karhunen-Loéve using Fourier Series Basis}

Assume $x(t)$ is periodic with period $NT$, where $N$ is the number of symbols and $T$ is the symbol period.

Both signal and noise are expanded using the Fourier Series Basis functions (complex exponentials). 
\begin{align} \label{fourier-series}
& x(t) = \sum_{m=-M}^M x_me^{j2\pi n t/(NT)} \\
& w(t) = \sum_{m=-M}^M w_me^{j2\pi n t/(NT)} \\
\end{align}
where $L$ is large enough to account for great part of the energy in $x(t)$. This method sacrifices efficiency for simplicity. Other methods can expand the noise and signal in different bases and thus reduce the dimensionality required \cite{forestieri}.

\begin{align} \nonumber
y(t_k) = (x_k + n)^HK(x_k + n)
\end{align}
where
\begin{equation}
K_{ij} = K\bigg(\frac{i-M-1}{NT}, \frac{j-M-1}{NT}\bigg)  = K(f_i, f_j) = H_o(f_i)H_e(f_i-f_j)H_o^*(f_i);
\end{equation}
$(\cdot)^H$ denotes Hermitian transpose (conjugate transpose); $n$ is a column vector of zero-mean Gaussian random variables with variance $N_0/N$, since it corresponds to the Fourier Series coefficients of the periodic extension of the ASE noise; and $x_k$ is a column vector given by

\begin{equation}
x_{i, k} = x_{i - M - 1}e^{j2\pi(i-M-1)t_k/(NT)}
\end{equation}
where $x_{i - M - 1}$ are the Fourier series coefficients of the periodic extension of period $NT$ of $x(t)$ \eqref{fourier-series}.

Let 
\begin{equation}
c_k = U^Hx_k
\end{equation}
where $K = U\Lambda U^H$. $K$ is Hermitian and thus it only has real eigenvalues.

Therefore,
\begin{equation}
y(t_k) = (c_k + z)^H\Lambda (c_k + z) = \sum_{i = 1}^{M_e} \lambda_i|c_{i, k}^2 + z_i|^2
\end{equation}

And the MGF is given by
\begin{equation} \label{ASE-MGF}
\Psi_y(s) = \prod_{i =1}^{M_e}\frac{1}{(1-\lambda_i\sigma^2s)^{N_{pol}}}\exp\bigg(\frac{\lambda_i|c_{i,k}|^2s}{1-\lambda_i\sigma^2s}\bigg)
\end{equation}
where $\sigma^2 = N_0/{NT}$

We can calculate the first two moments of $y$ using \eqref{ASE-MGF}
\begin{align}
& \mu_y = \frac{d}{ds}\Psi_y(s)\bigg|_{s=0} = \Psi_y(s)\frac{d}{ds}\log\Psi_y(s)\bigg|_{s=0} = \sum_{i=1}^{M_e} \lambda_i(N_{pol}\sigma^2 + |x_n|^2) \\
& \sigma_{y}^2 = \frac{d^2}{ds^2}\Psi_y(s)\bigg|_{s=0} -\mu_y^2 = \Psi_y(s)\frac{d^2}{ds^2}\log\Psi_y(s)\bigg|_{s=0} = \sum_{i =1}^{M_e} N_{pol}(\lambda_i\sigma^2)^2 + 2\sigma^2\lambda_i^2|c_{i,k}|^2
\end{align}
the derivatives are calculated with respect to the derivatives of $\log\Psi_y(s)$, which is easier to obtain, as calculated in \eqref{mgf-pdf-derivatives}. Thus the conditional probabilities of $y$ can be approximated as a Gaussian distribution such that $p(y|a_i) \sim \mathcal{N}(\mu_y, \sigma_y^2)$.

\subsection{pdf and tail probabilities including thermal noise and using the saddlepoint approximation}

Including thermal noise in \eqref{ASE-MGF}
\begin{equation}
\Psi_y(s) = \bigg[\prod_{i =1}^{M_e}\frac{1}{(1-\lambda_i\sigma^2s)^{N_{pol}}}\exp\bigg(\frac{\lambda_i|c_{i,k}|^2s}{1-\lambda_i\sigma^2s}\bigg)\bigg]e^{\frac{1}{2}\sigma_{T}^2s^2}
\end{equation}

\subsubsection{pdf}
From the inversion equation we can write
\begin{align} \label{mgf-pdf-derivatives}
& K(s,x) = \log (\Psi(s)) - sx = \sum_{i = 1}^{M_e}\bigg( -N_{pol}\log(1-\lambda_i\sigma^2s) + \frac{\lambda_i|c_{i,k}|^2s}{1-\lambda_i\sigma^2s}\bigg) +\frac{1}{2}\sigma_T^2s^2 - sx  \\ \nonumber
&\frac{\partial K(s,x)}{\partial s} = \sum_{i = 1}^{M_e}\bigg( N_{pol}\frac{\lambda_i\sigma^2}{1-\lambda_i\sigma^2s} + \frac{\lambda_i|c_{i,k}|^2s}{(1-\lambda_i\sigma^2s)^2}\bigg) + \sigma_T^2s - x \\ \nonumber
& = \sum_{i = 1}^{M_e}\bigg( \frac{N_{pol}\lambda_i\sigma^2(1-\lambda_i\sigma^2s) + \lambda_i^2|c_{i,k}|^2s}{(1-\lambda_i\sigma^2s)^2}\bigg) + \sigma_T^2s - x \\
& = \sum_{i = 1}^{M_e}\bigg( \lambda_i\frac{N_{pol}\sigma^2(1-\lambda_i\sigma^2s) +|c_{i,k}|^2s}{(1-\lambda_i\sigma^2s)^2}\bigg) + \sigma_T^2s - x \\
&\frac{\partial^2 K(s,x)}{\partial s^2} = \sum_{i = 1}^{M_e}\bigg( \frac{N_{pol}(\lambda_i\sigma^2)^2}{(1-\lambda_i\sigma^2s)^2} + \frac{2\sigma^2\lambda_i^2|c_{i,k}|^2}{(1-\lambda_i\sigma^2s)^3}\bigg) + \sigma_T^2
\end{align}

\subsubsection{tail}
From the inversion equation we can write 
\begin{align}
& K(s,x) = \log \Big(\frac{\Psi(s)}{s}\Big) - sx =  \sum_{i = 1}^{M_e}\bigg( -N_{pol}\log(1-\lambda_i\sigma^2s) + \frac{\lambda_i|c_{i,k}|^2s}{1-\lambda_i\sigma^2s}\bigg) -\log |s| +\frac{1}{2}\sigma_T^2s^2 - sx  \\
&\frac{\partial K(s,x)}{\partial s} = \sum_{i = 1}^{M_e}\bigg( \lambda_i\frac{N_{pol}\sigma^2(1-\lambda_i\sigma^2s) +|c_{i,k}|^2s}{(1-\lambda_i\sigma^2s)^2}\bigg) - \frac{1}{s} + \sigma_T^2s - x \\
&\frac{\partial^2 K(s,x)}{\partial s^2} = \sum_{i = 1}^{M_e}\bigg( \frac{N_{pol}(\lambda_i\sigma^2)^2}{(1-\lambda_i\sigma^2s)^2} + \frac{2\sigma^2\lambda_i^2|c_{i,k}|^2}{(1-\lambda_i\sigma^2s)^3}\bigg) + \frac{1}{s^2} + \sigma_T^2
\end{align}
\textbf{Note that $K(s, x)$ is different from the pdf calculation.}


\begin{thebibliography}{9}
	
	\bibitem{ber-saddlepoint-approx} Carl Helstrom; ``Performance Analysis of Optical Receivers by the Saddlepoint Approximation,'' \emph{IEEE Transactions on Communications}, 1979.
		
	\bibitem{agilent-RIN-measurement} Product Note 86100-7; ``Digital Communication Analyzer (DCA), Measure Relative Intensity Noise (RIN),'' Agilent Technologies. 
	
	\bibitem{forestieri} E. Forestieri, M. Secondini; ``On the error probability evaluation in lightwave systems with optical amplificaiton,'' \emph{Journal of Lightwave Technol.}, 2009.
	
\end{thebibliography}

\end{document}